\documentclass[12pt]{amsart}
\setlength{\parskip}{.1in}
\setlength{\parindent}{0cm}
%myalterations
\usepackage{amssymb}
\usepackage[usenames,dvipsnames,svgnames,table]{xcolor}
\usepackage[colorlinks=true,urlcolor=blue,pdfborder={0 0 .5}pdfnewwindow=true]{hyperref}
\usepackage{enumitem}
%\usepackage{amsthm}
\usepackage{graphicx}
\usepackage{verbatim}
\usepackage{tabularx}
%\usepackage{arydshln}
\usepackage{leftidx}
\usepackage{mathtools}
\usepackage{bm}
\usepackage{tikz-cd}
\usepackage{hyperref}
\usepackage{bm}

%\setlength{\dashlinedash}{.4pt}
%\setlength{\dashlinegap}{.8pt}
%\usepackage{amsthm}
\usepackage{verbatim}
%\usepackage{commath}
%My commands
%environment abbreviations
\newcommand{\bem}{\begin{matrix}}
\newcommand{\emm}{\end{matrix}}
\newcommand{\besm}{\begin{smallmatrix}}
\newcommand{\esm}{\end{smallmatrix}}
\newcommand{\benu}{\begin{enumerate}}
\newcommand{\eenu}{\end{enumerate}}
\newcommand{\bed}{\begin{description}}
\newcommand{\ed}{\end{description}}
\theoremstyle{definition}
\newtheorem{theorem}{Theorem}
\newtheorem{notation}[theorem]{Notation}
\newcommand{\bnot}{\begin{notation}}
\newcommand{\enot}{\end{notation}}
\newcommand{\bet}{\begin{theorem}}
\newcommand{\et}{\end{theorem}}
\newtheorem{axiom}[theorem]{Axiom}
\newcommand{\baxi}{\begin{axiom}}
\newcommand{\axi}{\end{axiom}}
\newtheorem{lemma}[theorem]{Lemma}
\newcommand{\bel}{\begin{lemma}}
\newcommand{\el}{\end{lemma}}
\newtheorem{corollary}[theorem]{Corollary}
\newcommand{\bec}{\begin{corollary}}
\newcommand{\ec}{\end{corollary}}
\newtheorem{observation}[theorem]{Observation}
\newcommand{\bo}{\begin{observation}}
\newcommand{\eo}{\end{observation}}
\newtheorem{exercise}[theorem]{Exercise}
\newcommand{\bex}{\begin{exercise}}
\newcommand{\ex}{\end{exercise}}

\newtheorem{definition}[theorem]{Definition}
\newcommand{\bdf}{\begin{definition}}
\newcommand{\edf}{\end{definition}}
\newtheorem{example}[theorem]{Example}
\newcommand{\bax}{\begin{example}}
\newcommand{\ax}{\end{example}}
\newcommand{\pru}{{ \bfseries \textcolor{red}{Proof:} }}

\newtheorem*{und}{Definition}
%symbol definitions
\newcommand{\un}[1]{\underline{#1}}
\newcommand{\mbZ}{\mathbb{Z}}
\newcommand{\mbR}{\mathbb{R}}
\newcommand{\mbN}{\mathbb{N}}
\newcommand{\mbQ}{\mathbb{Q}}
\newcommand{\mbC}{\mathbb{C}}
\newcommand{\mbF}{\mathbb{F}}
\newcommand{\mcS}{\mathcal{S}}
\newcommand{\mcP}{\mathcal{P}}
\newcommand{\hra}{\hookrightarrow}
\newcommand{\tra}{\twoheadrightarrow}
\newcommand{\lra}{\leftrightarrow}

\newcommand{\Ra}{\Rightarrow}
\newcommand{\mb}[1]{\mathbb{#1}}
\newcommand{\mc}[1]{\mathcal{#1}}
\newcommand{\bfs}[1]{{\bfseries #1}}
\newcommand{\bs}[1]{\boldsymbol{#1}}
%Operator definitions
\DeclareMathOperator{\Irr}{Irr}
\DeclareMathOperator{\triv}{triv}
\DeclareMathOperator{\cyc}{cyc}
\DeclareMathOperator{\lcm}{lcm}
\DeclareMathOperator{\expo}{x}
\DeclareMathOperator{\ord}{o}
\DeclareMathOperator{\imm}{im}
\DeclareMathOperator{\sgn}{sgn}
\DeclareMathOperator{\Sym}{Sym}
\DeclareMathOperator{\alt}{alt}
\DeclareMathOperator{\irr}{irr}
\DeclareMathOperator{\eqt}{Equiv}
\DeclareMathOperator{\pat}{Part}
%\DeclareMathOperator{\sgn}{sgn}
%\DeclareMathOperator{\Aut}{Aut}
\DeclareMathOperator{\Gl}{Gl}
\DeclareMathOperator{\M}{M}
\DeclareMathOperator{\Id}{Id}
\DeclareMathOperator{\fixx}{Fix}
\DeclareMathOperator{\suppp}{Supp}
\DeclareMathOperator{\gl}{Gl}
\DeclareMathOperator{\id}{Id}
\DeclareMathOperator{\Aut}{Aut}
\DeclareMathOperator{\Inn}{Inn}
\DeclareMathOperator{\orb}{orb}
\DeclareMathOperator{\ii}{I}
\DeclareMathOperator{\im}{im}
\DeclareMathOperator{\Fix}{Fix}
\DeclareMathOperator{\Co}{Co}
\DeclareMathOperator{\md}{md}
\DeclareMathOperator{\qt}{qt}
\DeclareMathOperator{\ExtendedGCD}{ExtendedGCD}
\DeclareMathOperator{\Mod}{Mod}
\DeclareMathOperator{\GCD}{GCD}
\newcommand{\nms}{\negmedspace}
\newcommand{\nts}{\negthinspace}


\newcommand{\itep}{\item {\bfseries Problem}\ }
\newcommand{\gen}[1]{\langle #1 \rangle}
\newcommand{\quot}[2]{#1/ #2}
\newcommand{\order}[1]{\left|<\nts #1 \nts s>\right|}

%These next two commands are for making answers. 
\newcommand{\beans}{\begin{description} \item[{ \bfseries \textcolor{red}{Answer}}]\ }
\newcommand{\eans }{\end{description}}
%\newcommand{\begin{comment}ex}{{ \bfseries \textcolor{red}{Answer}}}

%To turn the answer into problem sets use replace to replace \begin{comment} with \begin{comment} and \\end{comment}  by \end{comment}.
\newcommand{\lieb}[3][{{}}]{\frac{d^#1 #2}{d\,#3^#1}}
\begin{document} 

\begin{enumerate}


\item
\benu
\item Compute the addition and multiplication tables for $Z_5$.

The table for addition:
\begin{equation*}
\begin{array}{c|ccccc}

+ & 0 & 1 & 2 & 3 & 4 \\
\hline
0 & 0 & 1 & 2 & 3 & 4 \\
1 & 1 & 2 & 3 & 4 & 0 \\
2 & 2 & 3 & 4 & 0 & 1 \\
3 & 3 & 4 & 0 & 1 & 2 \\
4 & 4 & 0 & 1 & 2 & 3 \\

\end{array}
\end{equation*}


The table for multiplication
\begin{equation*}
\begin{array}{c|ccccc}

\cdot & 0 & 1 & 2 & 3 & 4 \\
\hline
0 & 0 & 0 & 0 & 0 & 0 \\
1 & 0 & 1 & 2 & 3 & 4 \\
2 & 0 & 2 & 4 & 1 & 3 \\
3 & 0 & 3 & 1 & 4 & 2 \\
4 & 0 & 4 & 3 & 2 & 1 \\

\end{array}
\end{equation*}


\item Compute the addition and multiplication tables for $Z_6$

The table for addition:
\begin{equation*}
\begin{array}{c|cccccc}

+ & 0 & 1 & 2 & 3 & 4 & 5 \\
\hline
0 & 0 & 1 & 2 & 3 & 4 & 5 \\
1 & 1 & 2 & 3 & 4 & 5 & 0 \\
2 & 2 & 3 & 4 & 5 & 0 & 1 \\
3 & 3 & 4 & 5 & 0 & 1 & 2 \\
4 & 4 & 5 & 0 & 1 & 2 & 3 \\
5 & 5 & 0 & 1 & 2 & 3 & 4 \\

\end{array}
\end{equation*}


The table for multiplication
\begin{equation*}
\begin{array}{c|cccccc}

\cdot & 0 & 1 & 2 & 3 & 4 & 5 \\
\hline
0 & 0 & 0 & 0 & 0 & 0 & 0 \\
1 & 0 & 1 & 2 & 3 & 4 & 5 \\
2 & 0 & 2 & 4 & 0 & 2 & 4 \\
3 & 0 & 3 & 0 & 3 & 0 & 3 \\
4 & 0 & 4 & 2 & 0 & 4 & 2 \\
5 & 0 & 5 & 4 & 3 & 2 & 1 \\

\end{array}
\end{equation*}

\eenu

\newpage %q2
\item Find all of the solutions to $567x\equiv 238 \mod 665$. Check your results.

\newpage %q3


\item Prove the Multi-Modulus Chinese Remainder theorem.
 
\benu 
\item Refer to a result in Problem set 2 that shows $(n_i,P_i)=1$. Thus for each $i$ we can find $\lambda_i,\omega_i$ such that $\lambda_in_i+\omega_iP_i=1$
\\
\\
This follows from Problem Set 2 Problem 3a
\\
\item
Take $x=\sum_{i=1}^na_i\omega_iP_i$ show that for all $i, x\equiv_{n_i}a_i$.
\\\\
For all $i$ we have  $a_i\omega_iP_i\equiv_{n_i} a_i$ and for $j\neq i,n_j|P_i$ so $a_i\omega P_i\equiv_{n_j}0$. Hence for all $i$, $a_i\omega_{i}P_i+\sum_{j\neq i}a_j\omega_jP_j=x\equiv_{n_i} a_i$.
\\
\item Refer to a result in Problem set 2 that shows that if for all $i$ ,$x\equiv_{n_i}a_i$ and $y\equiv_{n_i} a_i$ then $x\equiv_P y$
\\\\
Suppose that for all $i$ we have $x\equiv_{n_i}a_i$ and $y\equiv_{n_i} a_i$. Then for all $i, x-y\equiv_{n_i}0$. That is for all $i$ we have $n_i|(x-y)$. Since the $n_i$ are pairwise relatively prime, by Problem set 2 Problem 2d $P|(x-y)$.

\eenu

\newpage %q4
\item 
Take
\begin{align*}
n_1=105,&a_1=27\\
n_2=121,&a_2=35,\\
n_3=64,&a_3=14\\
n_4=169,&a_4=82
\end{align*}
We have $105\cdot121\cdot 64\cdot 169=137417280$
Find $s \in Z_{137417280}$ such that 
\begin{align*} 
s &\equiv_{105} 27\\
s &\equiv_{121}35\\
s &\equiv_{64} 14\\
s&\equiv_{169}82
\end{align*}
\\
Show me the steps. You should get $107223822$.



\newpage %q5
\item 
\begin{enumerate}
\item Prove {\bfseries Unit Bijection}
\item Prove {\bfseries Multiplicativity of the Totient}
\item Why is the calculation of $\varphi(p^n)$ correct
\item Prove BB Corollary 3.5.6
\end{enumerate}

\newpage %q6

\item
\benu
\item What is the set $Z_{18}^*$
\item Make a table like that above for $Z_{18}^*$
\begin{equation*}
\begin{array}{c|c|c}
\text{element} & \text{powers} & \text{order}

\\
\hline
1 & \{1\} & 1
\\
\hline
2 & \{1, 2, 4, 8, 16, 14, 10\} & 7
\\
\hline
3 & \{1, 3, 9\} & 3
\\
\hline
4 & \{1, 4, 16, 10\} & 4
\\
\hline
5 & \{1, 5, 7, 17, 13, 11\} & 6
\\
\hline
6 & \{1, 6, 0\} & 3
\\
\hline
7 & \{1, 7, 13\} & 3
\\
\hline
8 & \{1, 8, 10\} & 3
\\
\hline
9 & \{1, 9\} & 2
\\
\hline
10 & \{1, 10\} & 2
\\
\hline
11 & \{1, 11, 13, 17, 7, 5\} & 6
\\
\hline
12 & \{1, 12, 0\} & 3
\\
\hline
13 & \{1, 13, 7\} & 3
\\
\hline
14 & \{1, 14, 16, 8, 4, 2, 10\} & 7
\\
\hline
15 & \{1, 15, 9\} & 3
\\
\hline
16 & \{1, 16, 4, 10\} & 4
\\
\hline
17 & \{1, 17\} & 2

\end{array}
\end{equation*}

\item What are the generators of $Z_{18}^*$
\eenu

\newpage %q7

\item
\begin{enumerate}
\item What is the set $Z_{12}^*$
\item Make a table like that above for $Z_{12}^*$. Notice that $Z_{12}^*$ does not have generators.
\begin{equation*}
\begin{array}{c|c|c}
\text{element} & \text{powers} & \text{order}

\\
\hline
1 & \{1\} & 1
\\
\hline
2 & \{1, 2, 4, 8\} & 4
\\
\hline
3 & \{1, 3, 9\} & 3
\\
\hline
4 & \{1, 4\} & 2
\\
\hline
5 & \{1, 5\} & 2
\\
\hline
6 & \{1, 6, 0\} & 3
\\
\hline
7 & \{1, 7\} & 2
\\
\hline
8 & \{1, 8, 4\} & 3
\\
\hline
9 & \{1, 9\} & 2
\\
\hline
10 & \{1, 10, 4\} & 3
\\
\hline
11 & \{1, 11\} & 2


\end{array}
\end{equation*}

\end{enumerate}


\end{enumerate}
\end{document}
