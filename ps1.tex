\documentclass[11pt]{article}
\usepackage{amsmath}
\usepackage{amssymb}
\title{\textbf{Math 311 - Problem Set 1}}
\author{Guy Matz}
\date{\today}
\begin{document}

\maketitle

\newpage

\begin{enumerate}  % begin numbers
\item Let $X = \phi \ or \ Y = \phi$

\begin{enumerate} % begin letters
\item What is the set $X \times Y$\\ \\
Since all mappings will be of the form $(x,\phi) \vee (\phi,y)$, $X \times Y = \phi$\\
\item Let $X = \phi$, what is the set $F(X,Y)?$
\begin{align*}
F(X,Y) &= \{f: f:X \to Y\}
\\     &= \{f: f:\phi \to Y\}
\\     &= \{ \phi \}
\end{align*}

\item Let $X \neq \phi$ and $Y = \phi$, what is the set $F(X,Y)?$\\ \\
Let $f \in F$ be a function that maps $X \to Y$.  Since all mappings will be of the form $(x,\phi)$, no $f$ satifies the definition of a function, therefore $F(X,Y) = \phi$  
\end{enumerate} % end letters

\newpage

% Q2
\item

\begin{enumerate} % begin letters
\item Let $X, Y \neq \phi$ and suppose that every element of $F(X,Y)$ is injective.  Show that $|X| = 1$.
\\ \\
A function is defined as follows (From BB, p. 50):\\
Let $S$ and $T$ be sets. A \emph{function} from $S$ to $T$ is a subset $F$ of $S \times T$ such that for each element $x \in S$ there is exactly one element  $y \in T$ such that $(x,y) \in F$
\\ \\
Case: 
\\
\item Let $X \neq \phi \neq Y$ and suppose that every element of $F(X,Y)$ is surjective.  Show that $|Y| = 1$.
\\ \\
The only subset of $Y$ that guarantees the "every element of $F(X,Y)$ is surjective" is a singleton, since it is the only subset that will use every element in its co-domain
\\

%\begin{align*}

%\end{align*}
\end{enumerate} % end letters
\newpage

\item % problem 3
\begin{enumerate}
\item Let $A_0 , A_1 \subseteq X$. Show that $f_*(A_0 \cup A_1 ) = f_*(A_0) \cup f_*(A_1)$
\\ \\
Let $x \in (A_1 \cup A_2)$, wlog $x \in A_1$.  By definition, $f(x) \in f_*(A_1) \subseteq f_*(A_1) \cup f_*(A_2).$  We then conclude $f_*(A_1 \cup A_2) \subseteq f_*(A_1) \cup f_*(A_2)$.  Conversely, suppose $y \in f_*(A_1) \cup f_*(A_2)$, wlog $y \in f_*(A_1)$.  By definition, $y \in f_*(A_1 \cup A_2)$.  We can conclude $f_*(A_1) \cup f_*(A_2) \subseteq f_*(A_1 \cup A_2)$.
\\

\item It is obvious that $f_∗(A_0 \cap A_1) \subseteq f_∗(A_0) \cap f_∗(A_1)$.  Give an
example of a set $X$, non-empty subsets $A_1$ and $A_2$ of $X$ and
$f : X \to Y$ such that $f_∗(A_0 \cap A_1) = f_∗(A_0) \cap f_∗(A_1)$. Hint:
Take $X = \{1, 2, 3\}$ and $Y = \{1, 2\}$.
\\ \\
Let $f: X \to Y$ be defined by $f: \{1,2,3\} \to \{1,2\}$
\begin{align*}
f(1) &= 1\\
f(2) &= 2\\
f(3) &= 2\\
A_0 &= \{1,2\}\\
A_1 &= \{1,3\}\\
f_*(A_0) &= \{1,2\}\\
f_*(A_1) &= \{1,2\}\\
A_0 \cap A_1 &= \{1\}\\
f_*(1) &= \{1\}\\
f(A_0) \cap f(A_1) &= \{1\}
\end{align*}
\item Show that if $f:X \to Y$ is not surjective then for any subset $\phi \neq A,\   \overline{f_∗(A)} \neq f_∗(\overline{A})$.\\
\\
$A \subseteq X, f_*(A) \subseteq f_*(X)$\\
$\overline{f_*(X)} \subseteq \overline{f_*(A)}$\\
however,\\
$\overline{f_*(X)} \cap f_*(X) = \phi$\\
% Do I need this here?  If so, then what!?
% $\overline{f_*(X)} \cap f_*(something . . .?)$\\
$f_*(\overline{A}) \subseteq f_*(X)$\\
$f_*(\overline{A}) \nsubseteq \overline{f_*(A)}$\\
\\
\item Give an example of a surjection $f: X \to Y$ and $A \subseteq X$ such that $\overline{f_*(A)} \neq f_*(\overline{A})$\\
\\
\\
\begin{align*}
f_*(\{1,2\}) &= \{1,2\}\\
f_*(1,2) &= \phi\\
f_*(1,2) &= f_*(3) = 3
\end{align*}
\end{enumerate} % end of 3
\newpage

 % problem 4
\item Investigate $f^*\mathcal{P}(Y) \to \mathcal{P}(X)$. State and prove results
parallel to those of Problem 3. Hint: The results for intersections, and complements are different.\\

\begin{enumerate}
\item Let $A_0 , A_1 \subseteq \mathcal{P}(Y)$. Show that $f^*(A_0 \cup A_1 ) = f^*(A_0) \cup f^*(A_1)$
\\ \\
Let $y \in (A_0 \cup A_1)$, wlog $y \in A_0$.  By definition, $f^{-1}(y) \in f^*(A_0) \subseteq f^*(A_0) \cup f^*(A_1).$  We then conclude $f^*(A_0 \cup A_1) \subseteq f^*(A_0) \cup f^*(A_1)$.  Conversely, suppose $x \in f^*(A_0) \cup f^*(A_1)$, wlog $x \in f^*(A_0)$.  By definition, $x \in f^*(A_0 \cup A_1)$.  We can conclude $f^*(A_0) \cup f^*(A_1) \subseteq f^*(A_0 \cup A_1)$.
\\

\item Let $A_0 , A_1 \subseteq Y$. Show that $f^*(A_0 \cap A_1 ) = f^*(A_0) \cap f^*(A_1)$
\\ \\
Let $y \in (A_0 \cap A_1)$, wlog $y \in A_0$.  By definition, $f^{-1}(y) \in f^*(A_0) \subseteq f^*(A_0) \cap f^*(A_1).$  We then conclude $f^*(A_0 \cap A_1) \subseteq f^*(A_0) \cap f^*(A_1)$.  Conversely, suppose $x \in f^*(A_0) \cap f^*(A_1)$, wlog $x \in f^*(A_0)$.  By definition, $x \in f^*(A_0 \cap A_1)$.  We can conclude $f^*(A_0) \cap f^*(A_1) \subseteq f^*(A_0 \cap A_1)$.
\\


\item Show that if $f^{-1}:Y \to X$ is not surjective then for any subset $\phi \neq A,\   \overline{f^*(A)} \neq f^*(\overline{A})$.\\
\\
1$A \subseteq Y, f^*(A) \subseteq f^*(Y)$\\
$\overline{f^*(Y)} \subseteq \overline{f^*(A)}$\\
however,\\
$\overline{f^*(Y)} \cap f^*(Y) = \phi$\\
% Do I need this here?  If so, then what!?
% $\overline{f^*(X)} \cap f^*(something . . .?)$\\
$f^*(\overline{A}) \subseteq f^*(Y)$\\
$f^*(\overline{A}) \nsubseteq \overline{f^*(A)}$\\
\\
\item Give an example of a surjection $f^{-1}: Y \to X$ and $A \subseteq Y$ such that $\overline{f^*(A)} \neq f^*(\overline{A})$\\
\\
$f^*(\{1,2\}) = \{1,2\}$\\
$f^*(1,2) = \phi$\\
$f^*(1,2) = f^*(3) = 3$
\end{enumerate} % end of 4
\newpage

\item Let $f:X \to Y$ and $X \neq \phi$
\begin{enumerate}
\item Show that $f$ is injective if and only if $f$ has a left inverse\\
$\Rightarrow$Assume $f$ is injective.\\
There is a function $g:B \to A$ such that $g \circ f = id_X$.  Let $d \in X$, and define $g$ on an input $y$ as:
\begin{enumerate}
\item If there exists some $x \in X$ with $f(x) = y$, then we will let $g(y) = x$  By Problem \#1 (?), there is at most one x.
\item If y is not in the image of $f$, then $g(y) = d$
So, for any $x$, $(g \circ)(x) = x$.  By definition of $g$, since $y = f(x)$ is in the image of $f, g(y)$ is defined by the first rule (above) to be $x$.  Thus $g \circ f = id_X$. 
\end{enumerate} % end of function def'
$\Leftarrow$Assume $f$ has a left inverse, $g$.\\
If $f(x) = f(y)$ then $g(f(x)) = g(f(y))$. By definitionof $g$, $x = g(f(x))$ and $g(f(y)) = y$, and so $x = g(f(x)) = g(f(y)) = y$\\
\item Show that if $f$ has a left inverse, then $f$ is surjective\\
\end{enumerate} % end of 5

\end{enumerate} % end numbers
\end{document}