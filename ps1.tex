\documentclass[11pt]{article}
\usepackage{amsmath}
\usepackage{amssymb}
\title{\textbf{Math 311 - Problem Set 1}}
\author{Guy Matz}
\date{\today}
\begin{document}

\maketitle

\newpage

\begin{enumerate}  % begin numbers
\item Let $X = \phi \ or \ Y = \phi$

\begin{enumerate} % begin letters
\item What is the set $X \times Y$
\begin{align*}
    X \times Y &= {(x,y) ; x \in X \ and\ y\in Y}
\\             &= {(\phi, y) ; y \in Y } + {(x, \phi) ; x \in X }
\\             &= \phi
\end{align*}
\item Let $X = \phi$, what is the set $F(X,Y)?$
\begin{align*}
    F(X,Y) &= \{f: f:X \to Y\}
\\         &= \{f: f:\phi \to Y\}
\\         &= \phi
\end{align*}

\item Let $X \neq \phi$ and $Y = \phi$, what is the set $F(X,Y)?$
\begin{align*}
    F(X,Y) &= \{f: f:X \to Y\}
\\         &= \{f: f:X \to \phi\}
\\         &= f:X \to \phi
\end{align*}
  
\end{enumerate} % end letters

\newpage

% Q2
\item

\begin{enumerate} % begin letters
\item Let $X, Y \neq \phi$ and suppose that every element of $F(X,Y)$ is injective.  Show that $|X| = 1$.
\\ \\
A function is defined as follows (From BB, p. 50):\\
Let $S$ and $T$ be sets. A \emph{function} from $S$ to $T$ is a subset $F$ of $S \times T$ such that for each element $x \in S$ there is exactly one element  $y \in T$ such that $(x,y) \in F$
\\ \\
The only subset of $X$ that guarantees the "every element of $F(X,Y)$ is injective" is a singleton, since it is the only subset that will use every element in its domain
%
\\
\item Let $X \neq \phi \neq Y$ and suppose that every element of $F(X,Y)$ is surjective.  Show that $|Y| = 1$.
\\ \\
The only subset of $Y$ that guarantees the "every element of $F(X,Y)$ is surjective" is a singleton, since it is the only subset that will use every element in its co-domain
\\

%\begin{align*}

%\end{align*}
\end{enumerate} % end letters

\item % problem 3
\begin{enumerate}
\item Let $A_0 , A_1 \subseteq X$. Show that $f_∗ (A_0 ∪ A_1 ) = f_∗ (A_0) \cup f ∗ (A_1)$
\\ \\
Let $x \in (A_1 \cup A_2)$, wlog $x \in A_1$.  By definition, $f(x) \in f_*(A_1) \subseteq f_*(A_1) \cup f_*(A_2).$  We then conclude $f_*(A_1 \cup A_2) \subseteq f_*(A_1) \cup f_*(A_2)$.  Conversely, suppose $y \in f_*((A_1) \cup f_*(A_2))$, wlog $y \in f_*(A_1)$.  By definition, ?????? . . . We now conclude $f_*(A_1) \cup f_*(A_2) \subseteq f_*(A_1 \cup A_2)$.
\\

\item It is obvious that $f_∗(A_0 \cap A_1) \subseteq f_∗(A_0) \cap f_∗(A_1)$.  Give an
example of a set $X$, non-empty subsets $A_1$ and $A_2$ of $X$ and
$f : X \to Y$ such that $f_∗(A_0 \cap A_1) = f_∗(A_0) \cap f_∗(A_1)$. Hint:
Take $X = \{1, 2, 3\}$ and $Y = \{1, 2\}$.
\\ \\
$f: \{1,2,3\} \to \{1,2\}$\\
$f(1) = 1$\\
$f(2) = 2$\\
$f(3) = 2$\\
$A_1 = \{1,2\}$\\
$A_2 = \{1,3\}$\\
$f_*(A_1) = \{1,2\}$\\
$f_*(A_2) = \{1,2\}$\\
$A_1 \cap A_2 = \{1\}, f_*(1) = \{1\}$\\

\item Show that if $f : X \to Y$ is not surjective then for any subset $\phi \neq A,\   \overline{f_∗(A)} \neq f_∗(\overline{A})$.\\
\\
$A \subseteq X, f_*(A) \subseteq f_*(X)$\\
$\overline{f_*(X)} \subseteq \overline{f_*(A)}$\\
however,\\
$\overline{f_*(X)} \cap f_*(X) = \phi$\\
$\overline{f_*(X)} \cap f_*(something . . .?)$\\
$f_*(\overline{A}) \subseteq f_*(X)$\\
$f_*(\overline{A}) \nsubseteq \overline{f_*(A)}$\\
\\
\item Give an example of a surjection $f: X \to Y$ and $A \subseteq X$ such that $\overline{f_*(A)} \neq f_*(\overline{A})$\\
\\
\\
$f_*(\{1,2\}) = \{1,2\}$\\
$f_*(1,2) = \phi$\\
$f_*(1,2) = f_*(3) = 3$
\end{enumerate}

\end{enumerate} % end numbers

\end{document}