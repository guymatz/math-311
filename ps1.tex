\documentclass[11pt]{article}
\usepackage{amsmath}
\usepackage{amssymb}
\title{\textbf{Math 311 - Problem Set 1}}
\author{Guy Matz}
\date{\today}
\begin{document}

\maketitle

\newpage

\begin{enumerate}  % begin numbers
\item Let $X = \phi \ or \ Y = \phi$

\begin{enumerate} % begin letters
\item What is the set $X \times Y$\\ \\
The results of the cross will have no first component and/or no second component, therefore, $X \times Y = \phi$\\
\item Let $X = \phi$, what is the set $F(X,Y)?$
\begin{align*}
F(X,Y) &= \{f: f:X \to Y\}
\\     &= \{f: f:\phi \to Y\}
\\     &= \{ \phi \}
\end{align*}

\item Let $X \neq \phi$ and $Y = \phi$, what is the set $F(X,Y)?$\\ \\
Let $f \in F$ be a function that maps $X \to Y$.  The resulting pairs of the mapping will have no second component, therefore, $X \times Y = \phi$\\
\end{enumerate} % end letters

\newpage

% Q2
\item

\begin{enumerate} % begin letters
\item Let $X, Y \neq \phi$ and suppose that every element of $F(X,Y)$ is injective.  Show that $|X| = 1$.
\\ \\
A function is defined as follows (From BB, p. 50):\\
Let $S$ and $T$ be sets. A \emph{function} from $S$ to $T$ is a subset $F$ of $S \times T$ such that for each element $x \in S$ there is exactly one element  $y \in T$ such that $(x,y) \in F$
\\ \\
Case: 
\\
\item Let $X \neq \phi \neq Y$ and suppose that every element of $F(X,Y)$ is surjective.  Show that $|Y| = 1$.
\\ \\
The only subset of $Y$ that guarantees the "every element of $F(X,Y)$ is surjective" is a singleton, since it is the only subset that will use every element in its co-domain
\\

%\begin{align*}

%\end{align*}
\end{enumerate} % end letters
\newpage

\item % problem 3
\begin{enumerate}
\item Let $A_0 , A_1 \subseteq X$. Show that $f_*(A_0 \cup A_1 ) = f_*(A_0) \cup f_*(A_1)$
\\ \\
Let $x \in (A_1 \cup A_2)$, wlog $x \in A_1$.  By definition, $f(x) \in f_*(A_1) \subseteq f_*(A_1) \cup f_*(A_2).$  We then conclude $f_*(A_1 \cup A_2) \subseteq f_*(A_1) \cup f_*(A_2)$.  Conversely, suppose $y \in f_*(A_1) \cup f_*(A_2)$, wlog $y \in f_*(A_1)$.  By definition, $y \in f_*(A_1 \cup A_2)$.  We can conclude $f_*(A_1) \cup f_*(A_2) \subseteq f_*(A_1 \cup A_2)$.
\\

\item It is obvious that $f_*(A_0 \cap A_1) \subseteq f_*(A_0) \cap f_*(A_1)$.  Give an example of a set $X$, non-empty subsets $A_1$ and $A_2$ of $X$ and
$f : X \to Y$ such that $f_*(A_0 \cap A_1) \neq f_*(A_0) \cap f_*(A_1)$. Hint:
Take $X = \{1, 2, 3\}$ and $Y = \{1, 2\}$.
\\ \\
Let $f: X \to Y$ be defined by $f: \{1,2,3\} \to \{1,2\}$ where\\
$$f(1) = 1, 
f(2) = 2, 
f(3) = 2$$
\begin{align*}
A_0 &= \{1,2\}\\
A_1 &= \{1,3\}\\
f_*(A_0) &= \{1,2\}\\
f_*(A_1) &= \{1,2\}\\
A_0 \cap A_1 &= \{1\}\\
f_*(A_0 \cap A_1) = f(1) &= \{1\}\\
f_*(A_0) \cap f_*(A_1) &= \{1,2\}
\end{align*}
Therefore $f_*(A_0 \cap A_1) \neq f_*(A_0) \cap f_*(A_1)$
\\
% 3c
\item Show that if $f:X \to Y$ is not surjective then for any subset $\phi \neq A, \overline{f_*(A)} \neq f_*(\overline{A})$.\\
\\
Since $f$ is not surjective ther is a $y_0 \in Y$ such that $y_0 \notin f_*(X)$.  Let $A \subseteq X$ be non-empty.  Since $y_0 \notin f_*(X), y_0 \notin f_*(\overline{A})$, which means that it must be in $\overline{f_*(A)}$, and, therefore, $f_*(\overline{A}) \neq \overline{f_*(A)}$
\\
\item Give an example of a surjection $f: X \to Y$ and $A \subseteq X$ such that $\overline{f_*(A)} \neq f_*(\overline{A})$\\
Let $f$ be defined as in 2b above and let $A \subset X$, where $A = \{1,2\}$
\begin{align*}
f_*(\{1,2\}) &= \{1,2\}\\
\overline{f_*(\{1,2\})} &= \phi\\
f_*(\overline{\{1,2\}}) &= f_*(\{3\}) = \{2\}
\end{align*}
\end{enumerate} % end of 3
\newpage

 % problem 4
\item Investigate $f^*\mathcal{P}(Y) \to \mathcal{P}(X)$. State and prove results
parallel to those of Problem 3. Hint: The results for intersections, and complements are different.\\

\begin{enumerate}
\item Let $A_0 , A_1 \subseteq \mathcal{P}(Y)$. Show that $f^*(A_0 \cup A_1 ) = f^*(A_0) \cup f^*(A_1)$
\\ \\
Let $y \in (A_0 \cup A_1)$, wlog $y \in A_0$.  By definition, $f^{-1}(y) \in f^*(A_0) \subseteq f^*(A_0) \cup f^*(A_1).$  We then conclude $f^*(A_0 \cup A_1) \subseteq f^*(A_0) \cup f^*(A_1)$.  Conversely, suppose $x \in f^*(A_0) \cup f^*(A_1)$, wlog $x \in f^*(A_0)$.  By definition, $x \in f^*(A_0 \cup A_1)$.  We can conclude $f^*(A_0) \cup f^*(A_1) \subseteq f^*(A_0 \cup A_1)$.
\\

\item Let $A_0 , A_1 \subseteq Y$. Show that $f^*(A_0 \cap A_1 ) = f^*(A_0) \cap f^*(A_1)$
\\ \\
Let $y \in (A_0 \cap A_1)$, wlog $y \in A_0$.  By definition, $f^{-1}(y) \in f^*(A_0) \subseteq f^*(A_0) \cap f^*(A_1).$  We then conclude $f^*(A_0 \cap A_1) \subseteq f^*(A_0) \cap f^*(A_1)$.  Conversely, suppose $x \in f^*(A_0) \cap f^*(A_1)$, wlog $x \in f^*(A_0)$.  By definition, $x \in f^*(A_0 \cap A_1)$.  We can conclude $f^*(A_0) \cap f^*(A_1) \subseteq f^*(A_0 \cap A_1)$.
\\


\item Show that if $f^{-1}:Y \to X$ is not surjective then for any subset $\phi \neq A,\   \overline{f^*(A)} \neq f^*(\overline{A})$.\\
\\
1$A \subseteq Y, f^*(A) \subseteq f^*(Y)$\\
$\overline{f^*(Y)} \subseteq \overline{f^*(A)}$\\
however,\\
$\overline{f^*(Y)} \cap f^*(Y) = \phi$\\
% Do I need this here?  If so, then what!?
% $\overline{f^*(X)} \cap f^*(something . . .?)$\\
$f^*(\overline{A}) \subseteq f^*(Y)$\\
$f^*(\overline{A}) \nsubseteq \overline{f^*(A)}$\\
\\
\item Give an example of a surjection $f^{-1}: Y \to X$ and $A \subseteq Y$ such that $\overline{f^*(A)} \neq f^*(\overline{A})$\\
\\
$f^*(\{1,2\}) = \{1,2\}$\\
$f^*(1,2) = \phi$\\
$f^*(1,2) = f^*(3) = 3$
\end{enumerate} % end of 4
\newpage

% Q5
\item Let $f:X \to Y$ and $X \neq \phi$
\begin{enumerate}
\item Show that $f$ is injective if and only if $f$ has a left inverse\\
$\Leftarrow$Assume $f$ has a left inverse.\\
Let $g : Y \to X$ such that $g \circ f = Id_X$ be the left inverse of $f$.  We want to show that $f$ in injective.  Let $x_1, X2 \in X$.  Then $f(x_1) = f(x_2)$, and $(g(f(x_1)) = g(f(x_2))$, so $x_1 = x_2$ which shows injectivity.\\\\
$\Rightarrow$Assume $f$ is injective\\
We want to find a left inverse, $g : Y \to X$, such that $g \circ f = Id_X$.
\item Show that if $f$ has a left inverse, then $f$ is surjective\\\\
Let $f$ be a left inverse $g : Y \to X$ such that  $g \circ f = Id_X$.  We want to show that $f$ is surjective, i.e. $\forall y \in Y, \exists x \in X$ such that $f(x) = y$
% MORE HERE
\\
\item Show that if $f$ has a left inverse $g_l$ and $f$ has a right inverse\\
$g_r$ then $g_l = g_r$. Use $(g_l \circ f) \circ g_r = g_l ◦ (f \circ g_r )$.\\
\\
$$(g_l \circ f) \circ g_r = g_l \circ (f \circ g_r)$$
$$Id_X \circ g_r = g_l \circ Id_X$$
$$g_r = g_l$$
\end{enumerate} % end of 5

\newpage
% Q6
\item \textbf{Lemma 1}: If $f:X \leftrightarrow Y$ is bijective then $f$ has a right inverse
\begin{enumerate}
\item Just refererring to this lemma and the results of Problem 5, give a proof of BB proposition 2.1.7.\\
\\
By 5a we have shown that if $f$ has a left inverse then it is injective.  Using the Lemma we may go on to say that if $f$ has \emph{any} inverse, then it is injective.  By 5b we know that $f$ is surjective.  Using the two results and the Lemma we can say that $f$ is bijective\\
\item Let $f: X_1 \leftrightarrow X_2$.  Show that $(f^{-1})^{-1} = f$\\\\
Let $x_1 \in X_1$.  Since $f$ is bijective we have that $f(x_1) = x_2 \in X_2$ and $f^{-1}(x_2) = x_1 \in X_1$.  Using composition we can say $f^{-1} \circ f^{-1}(x_2) = f^{-1}(f^{-1}(x_2)) = f^{-1}(x_1) = x_2 \in X_2$, which is our definition of $f$.
\end{enumerate}
% end Q6

\newpage
% Q7
\item
\begin{enumerate}
\item Show that if $f :X_1 \hookrightarrow X_2$ and $g :X_2 \hookrightarrow X_3$ then $g \circ f: X_1 \hookrightarrow X_3$\\\\
% Do I need more here?  I don't use the definition of injectivity . . .
Let $x_{1a},x_{1b} \in X1, x_{2a},x_{2b} \in X_2, x_{3a},x_{3b} \in X_3$.  Then,\\
$$g \circ f(x_{1a}) = g(f(x_{1a})) = g(x_{2a}) = x_{3a} \in X_3$$
% Now I guess I need to say that x_1a = x_1b, x_2a = x_2b and x_3a = x_3b?
\\

\item Formulate and prove the corresponding result for surjections.\\
Show that if $f :X_1 \twoheadrightarrow X_2$ and $g :X_2 \twoheadrightarrow X_3$ then $g \circ f: X_1 \twoheadrightarrow X_3$\\\\
\item Formulate the corresponding result for bijections. Prove it by simply referring to previous results.\\
Show that if $f :X_1 \leftrightarrow X_2$ and $g :X_2 \leftrightarrow X_3$ then $g \circ f: X_1 \leftrightarrow X_3$\\\\
\item Let $f :X_1 \leftrightarrow X_2$ and $g :X_2 \leftrightarrow X_3$.  Show that $(g \circ f)^{-1} = f^{-1} \circ g^{-1}$\\
\end{enumerate}
% end Q7


\newpage
% Q8
\item We denote the set $\{f \in F(X) : f \hookrightarrow X \to X\}$ by $O(X, X)$, and the set $\{f \in F(X, X) : f : X \twoheadrightarrow X$ by $T(X, X)$.  $S_X = O(X,X) \cap T(X,X)$ is the set of bijective maps from $X$ to $X$
\begin{enumerate}
\item What previous result shows that $O(X, X)$ is closed?\\\\
7a
\item What previous result shows that $T(X, X)$ is closed?\\\\
7b
\item What previous result shows that $S_X$ is closed?\\\\
7c
\end{enumerate}
% end Q8

\newpage
% Q9 
\item What result in BB shows that we have a function $\mathcal{I} : S_X \to S_X$ given by $s(f) = g$ such that $g \circ f = Id_X = f \circ g$? Equivalently $\mathcal{I}(f) = f^{-1}$ is a \emph{function}\\\\
We are looking for a result that shows that any $f \in S_x$ has a unique inverse.  Proposition 2.1.7 shows this result.
% end Q9

\newpage
% Q10 
\item Let $A$ and $B$ be cofinite subsets of $X$.\\
\begin{enumerate}
\item Show that $A \cup B$ is cofinite.\\\\
We need to show that $\overline{A \cup B}$ is finite.  $\overline{A \cup B} = \overline{A} \cap \overline{B}$. Since $A$ and $B$ are both cofinite subsets of $X$, so are both $\overline{A}$ and $\overline{B}$ cofinite.  The intersection of two finite sets is finite.
\item Show that $A \cap B$ is cofinite.\\\\
We need to show that $\overline{A \cap B}$ is finite.  $\overline{A \cap B} = \overline{A} \cup \overline{B}$. Since $A$ and $B$ are both cofinite subsets of $X$, so are both $\overline{A}$ and $\overline{B}$ cofinite.  The union of two finite sets is finite.
\end{enumerate}
% end Q10

\newpage
% Q11
\item
\begin{enumerate}
\item Show that Co$(X)$ is closed under composition.  Hint: Show that Fix$(f) \cap$ Fix$(g) \subseteq$ Fix($f \circ g$).\\
\\
$$Fix(f) = \{x\in X; f(x) = x\}$$
$$Co(X) = \{f : Fix(f)\ is\ cofinite\}$$
\\
We want to show that if $f,g \in Co(X)$, then $f \circ g \in Co(X)$\\
Fix($f$), Fix($g$) are cofinite.  By Problem 10, Fix($f$) $\cap$ Fix $g$ is cofinite, 
\\
\item Show that if $f \in S_X$ is cofinite, so is $f^{-1}$.
\end{enumerate}
% end Q11

\newpage
% Q12
\item Show that Co($X$) is conjugate invariant
\\
\\
Let $f \in $ Co($X$), $g \in S_X$.  We want to show that $g \circ f \circ g^{-1}$ is confinite.\\
% end Q12

\newpage
% Q13
\item Define a relation $\sim$ on $F(X, X)$ by $f \sim g$ if there is an $\omega \in S_X$ such that $f = \omega \circ g \circ \omega^{-1}$.  Show that $\sim$ is an equivalence relation.\\
\\
To be an equivalance relation, the relation $\sim$ must satisfy the following properties:
\begin{enumerate}
\item Reflexive Law; $f \sim f$\\
$f = \omega \circ f \circ \omega^{-1}$.  Let $\omega = Id_X,$ then $f = Id_X \circ f \circ Id_X^{-1} = f$ 
\item Symmetric Law; If $f \sim g$, then $g \sim f$\\
Let $f = \omega \circ g \circ \omega^{-1}$\\
Then $g = \omega^{-1} \circ f \circ \omega = (\omega^{-1}) \circ f \circ (\omega^{-1})^{-1}$
\item Transitive Law; If $f \sim g$ and $g \sim h$, then $f \sim h$\\
Let $f = \omega \circ g \circ \omega^{-1}$\\
and $g = \omega \circ h \circ \omega^{-1}$\\
Then $f = \omega \circ \omega \circ h \circ \omega^{-1} \circ \omega^{-1} = (\omega^2) \circ h \circ (\omega^2)^{-1}$
\end{enumerate}
% end Q13

\newpage
% Q14
\item Let $X$ be a set and define a relation $\sim$ on $\mathcal{P}(X)$ by $A \sim B$ if there exists a bijection $f:A \leftrightarrow B$.  Prove that $\sim$ is an equivalence relation.
% end Q14
\\\\
To be an equivalance relation, the relation $\sim$ must satisfy the following properties:
\begin{enumerate}
\item Reflexive Law; $A \sim A$\\
\\
Let $f = Id_X$, then $f(A) = A$\\
\item Symmetric Law; If $A \sim B$, then $B \sim A$\\
\\Let $f:A \to B$ where $f_*(A) = B$.  Since $f$ is bijective, there exists $f^{-1}$ such that 	$f^*(B) = A$
\\
\item Transitive Law; If $A \sim B$ and $B \sim C$, then $A \sim C$\\
\\
Let $f:A \leftrightarrow B$ and $g:B \leftrightarrow C$
\\
Since $f_*(A) = B$ and $g_*(B) = C$, and both $f$ and $g$ are bijective, we have $(g \circ f)(A) = C$
\end{enumerate}
\end{enumerate} % end numbers
\end{document}