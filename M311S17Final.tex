 \documentclass[12pt]{amsart}
\setlength{\parskip}{.1in}
\setlength{\parindent}{0cm}
%myalterations
\usepackage{amssymb}
\usepackage[usenames,dvipsnames,svgnames,table]{xcolor}
\usepackage[colorlinks=true,urlcolor=blue,pdfborder={0 0 .5}pdfnewwindow=true]{hyperref}
\usepackage{enumitem}
%\usepackage{amsthm}
\usepackage{graphicx}
\usepackage{verbatim}
\usepackage{tabularx}
\usepackage{arydshln,leftidx,mathtools}
\usepackage{bm}
\usepackage{tikz-cd}
\usepackage{hyperref}
\usepackage{bm}

\setlength{\dashlinedash}{.4pt}
\setlength{\dashlinegap}{.8pt}
%\usepackage{amsthm}
\usepackage{verbatim}
%\usepackage{commath}
%My commands
%environment abbreviations
\newcommand{\bem}{\begin{matrix}}
\newcommand{\emm}{\end{matrix}}
\newcommand{\besm}{\begin{smallmatrix}}
\newcommand{\esm}{\end{smallmatrix}}
\newcommand{\benu}{\begin{enumerate}}
\newcommand{\eenu}{\end{enumerate}}
\newcommand{\bed}{\begin{description}}
\newcommand{\ed}{\end{description}}
\theoremstyle{definition}
\newtheorem{theorem}{Theorem}
\newtheorem{notation}[theorem]{Notation}
\newcommand{\bnot}{\begin{notation}}
\newcommand{\enot}{\end{notation}}
\newcommand{\bet}{\begin{theorem}}
\newcommand{\et}{\end{theorem}}
\newtheorem{axiom}[theorem]{Axiom}
\newcommand{\baxi}{\begin{axiom}}
\newcommand{\axi}{\end{axiom}}
\newtheorem{lemma}[theorem]{Lemma}
\newcommand{\bel}{\begin{lemma}}
\newcommand{\el}{\end{lemma}}
\newtheorem{corollary}[theorem]{Corollary}
\newcommand{\bec}{\begin{corollary}}
\newcommand{\ec}{\end{corollary}}
\newtheorem{observation}[theorem]{Observation}
\newcommand{\bo}{\begin{observation}}
\newcommand{\eo}{\end{observation}}
\newtheorem{exercise}[theorem]{Exercise}
\newcommand{\bex}{\begin{exercise}}
\newcommand{\ex}{\end{exercise}}

\newtheorem{definition}[theorem]{Definition}
\newcommand{\bdf}{\begin{definition}}
\newcommand{\edf}{\end{definition}}
\newtheorem{example}[theorem]{Example}
\newcommand{\bax}{\begin{example}}
\newcommand{\ax}{\end{example}}
\newcommand{\pru}{{ \bfseries \textcolor{red}{Proof:} }}

\newtheorem*{und}{Definition}
%symbol definitions
\newcommand{\un}[1]{\underline{#1}}
\newcommand{\mbZ}{\mathbb{Z}}
\newcommand{\mbR}{\mathbb{R}}
\newcommand{\mbN}{\mathbb{N}}
\newcommand{\mbQ}{\mathbb{Q}}
\newcommand{\mbC}{\mathbb{C}}
\newcommand{\mbF}{\mathbb{F}}
\newcommand{\mcS}{\mathcal{S}}
\newcommand{\mcP}{\mathcal{P}}
\newcommand{\hra}{\hookrightarrow}
\newcommand{\tra}{\twoheadrightarrow}
\newcommand{\lra}{\leftrightarrow}
\newcommand{\ep}{\epsilon}
\newcommand{\Ra}{\Rightarrow}
\newcommand{\mb}[1]{\mathbb{#1}}
\newcommand{\mc}[1]{\mathcal{#1}}
\newcommand{\bfs}[1]{{\bfseries #1}}
\newcommand{\bs}[1]{\boldsymbol{#1}}
%Operator definitions
\DeclareMathOperator{\Irr}{Irr}
\DeclareMathOperator{\triv}{triv}
\DeclareMathOperator{\cyc}{cyc}
\DeclareMathOperator{\lcm}{lcm}
\DeclareMathOperator{\expo}{x}
\DeclareMathOperator{\ord}{o}
\DeclareMathOperator{\imm}{im}
\DeclareMathOperator{\sgn}{sgn}
\DeclareMathOperator{\Sym}{Sym}
\DeclareMathOperator{\alt}{alt}
\DeclareMathOperator{\irr}{irr}
\DeclareMathOperator{\eqt}{Equiv}
\DeclareMathOperator{\pat}{Part}
%\DeclareMathOperator{\sgn}{sgn}
%\DeclareMathOperator{\Aut}{Aut}
\DeclareMathOperator{\Gl}{Gl}
\DeclareMathOperator{\M}{M}
\DeclareMathOperator{\Id}{Id}
\DeclareMathOperator{\fixx}{Fix}
\DeclareMathOperator{\suppp}{Supp}
\DeclareMathOperator{\gl}{Gl}
\DeclareMathOperator{\id}{Id}
\DeclareMathOperator{\Aut}{Aut}
\DeclareMathOperator{\Inn}{Inn}
\DeclareMathOperator{\orb}{orb}
\DeclareMathOperator{\ii}{I}
\DeclareMathOperator{\im}{im}
\DeclareMathOperator{\Fix}{Fix}
\DeclareMathOperator{\Co}{Co}
\DeclareMathOperator{\md}{md}
\DeclareMathOperator{\qt}{qt}
\DeclareMathOperator{\ExtendedGCD}{ExtendedGCD}
\DeclareMathOperator{\Mod}{Mod}
\DeclareMathOperator{\GCD}{GCD}
\newcommand{\nms}{\negmedspace}
\newcommand{\nts}{\negthinspace}


\newcommand{\itep}{\item {\bfseries Problem}\ }
\newcommand{\gen}[1]{\langle \nts#1 \nts\rangle}
\newcommand{\quot}[2]{#1/ #2}
\newcommand{\order}[1]{\left|<\nts #1 \nts s>\right|}

%These next two commands are for making answers. 
\newcommand{\beans}{\begin{description} \item[{ \bfseries \textcolor{red}{Answer}}]\ }
\newcommand{\eans }{\end{description}}
%\newcommand{\begin{comment}ex}{{ \bfseries \textcolor{red}{Answer}}}

%To turn the answer into problem sets use replace to replace \begin{comment} with \begin{comment} and \\end{comment}  by \end{comment}.
\newcommand{\lieb}[3][{{}}]{\frac{d^#1 #2}{d\,#3^#1}}

\title{\textbf{Math 311 - Final}}
\author{Guy Matz}
\date{\today}

\begin{document} 

%\maketitle
%\newpage % Q1

\begin{enumerate}[series=p]
\itep 
\label{drp}Let $m>1$ and suppose that $m=ab=a'b'$ with $a,b,a'$ and $b'$ greater than $0$ and $(a,b')=(a',b)=1$. Show that  $a=a'$ and $b=b'$

\newpage

\itep
Let $G,\ast$ be a group and $S,\circ$ be binary structure. Suppose $\psi\colon G\to S$ is a surjective homomorphism, that is $\psi(a\ast b)=\psi(a)\circ\psi(b)$
\benu
\item Prove that $S,\circ$ is a group.
\item Let $K=\psi^{-1}(\psi(e_G))$. Show that $K\leq G$
\eenu
\end{enumerate}

\newpage

\begin{enumerate}[resume=p]
\itep \label{D4comp} Use BB Table 3.6.1 to do the computations.
\benu
\item (Double Transpositions) \label{dt} In $A_4$ we have the set $T_4=\{Id,(1,2)(3,4),(1,3)(2,4),(1,4)(2,3)\}$. Complete the following table:

\begin{equation*}
\begin{array}{c|c|c|c|c}
	& \Id	&(1,2)(3,4)&(1,3)(2,4)&(1,4)(2,3)\\
\hline
\Id	&	&	&	&\\
\hline
(1,2)(3,4)	&	&	&	&\\
\hline
(1,3)(2,4)&	&	&	&\\
\hline	
(1,4)(2,3)&	&	&	&
\end{array}				
\end{equation*}
\item What does this table tell you about $T_4$ in relation to $A_4$?
\item Set $a=(1,2)(3,4),b=(1,3)(2,4),c=(1,4)(2,3)$ and substitute these letters into the table you just got. What table in BB do you get?
\item ($D_4$) Complete the following table
\begin{equation*} 
\begin{array}{c|c|c|c|c}
	&\Id	&(1,2)(3,4)	&(1,3)(2,4)	&(1,4)(2,3)	 \\
\hline	
	\Id&	&	&	&	\\
\hline	
	(1,3)&	&	&	&
\end{array}
\end{equation*}

\eenu
\end{enumerate}

\newpage

\begin{enumerate}[resume=p]
\itep  Show that if $H\unlhd G$ then for all $g\in G, gHg^{-1}=H$.
\beans
Suppose that $H\unlhd G$ and $g\in G$ then $gHg^{-1}\subseteq H$. Then $H=g^{-1}(gHg^{-1})g\subseteq g^{-1}Hg$. Taking $k=g^{-1}$ we have for all $k\in G, H\subseteq kHk^{-1}$. Thus for all $g\in G,gHg^{-1}=g$

\eans

\newpage

\itep 
\benu
\item Let $H\leq G$. Show that if for all $g\in G$, $gH=Hg$ then $H\unlhd G$
\beans If $gH=Hg$ multiplying on the right by $g^{-1}$ we have $gHg^{-1}=H$
\eans
\item Let $H\unlhd G$ and $g\in G$. Show that $gH=Hg$.  
\beans 
Assume $H\unlhd G$ then for any $g\in G$ we have $g^{-1}Hg=H$. Hence 
\[gH=g(g^{-1}Hg)=eHg=Hg
\]
\eans
\eenu
\end{enumerate}

\newpage

\begin{enumerate}[resume=p]
\itep \label{mp} Let $\Phi$ be a multiplicative partition of $G$. For the rest of this problem we will assume the partition is $\Phi$ and we will denote its cells by $[g]$ rather than $[g]_\Phi$. 
\benu 
 \item  Show that $\pi_\Phi\colon G\to \Phi,\pi_\Phi(g)=[g]$ is a homomorphism of binary structures.
\item What result above shows that $\Phi$ is a group?
\item What result above shows that $[e]\leq G$.
\item What result above shows that $[e]\unlhd G$
\eenu
\end{enumerate}

\newpage

\begin{enumerate}[resume=p]
\itep 
\benu

 \item Show that if $H\unlhd G$ then $G/H$ is multiplicative.
\beans
Assume  $H\unlhd G$. Consider $g_0H,g_1H\in G/H$. We have $g_1Hg_1^{-1}=H$. Thus 
\[g_0Hg_1H=g_0(g_1Hg_1^{-1})g_1H=g_0g_1HeH=g_0g_1HH=g_0g_1H\in G/H
\]
\eans
\item Let $\Phi$ be a multiplicative partition of $G$. Show that $\Phi=G/[e]$. Hint: $G/[e]$ is the cells of the equivalence relation $a\sim_{[e]} b$ if $a^{-1}b\in [e]$. By definition $\Phi$ is the cells of the equivalence relation $\pi_\Phi$. Use Problem 6 to show these equivalence relations are the same.
\eenu
\end{enumerate}

\newpage


\begin{enumerate}[resume=p]
\itep Let $n>2$. Give an example of a group $G$ with a subgroup $H$ of index $n$ such that $H$ is not normal in $G$. You must prove that in your example $H$ is not a  normal subgroup, that is you must give subgroup of $H$ of $G$ and an element in $G$ such that $gHg^{-1}$ is not a subset of $H$. Hint: Consider $S_n$ the subset $\{\omega\in S_n:\omega(n)=n\}$ and the transposition $(1,n)$.

\end{enumerate}

\newpage

\begin{enumerate}[resume=p]
\itep Show that $N[H]$ is a subgroup of $G$.
\beans Take $a, b\in N[H]$. We have 
\[(ab^{-1})H(ab^{-1})^{-1}=a(b^{-1}Hb)a^{-1}=aHa^{-1}=H\]
\eans
\end{enumerate}

\newpage

\begin{enumerate}[resume=p]
\itep 
\benu  
\item Show that if $K\leq N[H]$ then $KH=HK$.
\beans
By the equality of left and right cosets, that is for $a\in N[H],aH=Ha$ we have 
\[KH=\bigcup_{k\in K}kH=\bigcup_{k\in K}Hk=HK
\]

\eans
\item Prove that if $K\leq N[H]$ then $KH\leq G$
\beans
\[ KH(KH)^{-1}=KHH^{-1}K^{-1}\\=KHK^{-1}=HKK^{-1}=HK=KH
\]
\eans
\item\label{quotiso} Let $H\lhd G$ and $K\leq G$ such that $K\cap H=\phi$ and $KH=G$. As in BB 3.7.7 we have the natural map $\pi_H\colon G\to G/H$ show that $\pi|_K\colon K\to G/H$ is an isomorphism.
\item \label{count} Let $G$ be a finite group, $H\unlhd K,K\leq N[H]$ and $H\cap K=G$. Show that $|HK|=|H||K|$
\eenu
\end{enumerate}

\newpage

\begin{enumerate}[resume=p]
\itep
\benu
\item Show that $KT_4=S_4$. (Just count sizes).
\item We have $\gen{(1,2,3)}T_4\leq A_4$. Use counting  to show that $\gen{(1,2,3)}T_4=A_4$.


\item We have $T_4\lhd D_4, \gen{(1,3)}<D_4$ and $\gen{(1,3)}\}\cap T_4=\{\Id\}$. Use counting to show that $\gen{(1,3)}T_4=D_4$.
\eenu
\end{enumerate}

\newpage

\begin{enumerate}[resume=p]
\itep (External Direct Product)
\benu
\item Let $h\in H,k\in K$ show that $hk=kh$.
\item Show that $H\cap K=\{e\}$.
\item Show that $H\unlhd G$ and $K\unlhd G$.
\item Show that $HK=G$
\eenu
\end{enumerate}

\newpage

\begin{enumerate}[resume=p]
\itep (Internal Direct Product)Let $H\leq G$,$K\leq G$. We have the map $\phi\colon H\times K\to G,\phi((h.k))=hk$. It is usually not a homomorphism
\benu
\item Given $H\leq G$ and $K\leq G$ we have $p\colon H\times K\to G,p(hk)=hk$. This map is by definition surjective. Show that it is injective if and only if $H\cap K=\{e\}$.
\item Also assume that $H\unlhd G$ and $K\unlhd G$ show that $\phi$ is a homomorphism. Hint: Look at the commutator $[h,k]$ for $h\in H$ and $k\in K$.
\eenu
\end{enumerate}

\newpage


\begin{enumerate}[resume=p]
\itep Assuming The Theorem below, prove the Corollary
\bet[Cauchy] \label{ct} Let $G$ be a finite group and $p$ be a prime such that $p$ divides the order of $G$ then $G$ has an element of order $p$.
\et
\bec \label{corpgroup} Let $G$ be a finite group and suppose the order of any element of $G$ is a power of $p$, then $G$ is a $p$-group
\ec
\end{enumerate} 

\newpage

\begin{enumerate}[resume=p]
\itep \label{startdecomp} Let $|G|=mn$ 
\benu
\item
 Show that $I_m\leq K_n$. 
 
\item Now assume $(m,n)=1$ so that  that there are solutions in integers to $\lambda m+\omega n=1$.
\benu
\item Show that $K_m\cap K_n=\{e\}$
\item Show that $I_mI_n=G$.
\item Show that $p:K_n\times K_m\to G, p((k_1,k_2))= k_1k_2$ is an isomorphism
\eenu
\eenu

\end{enumerate}

\newpage

\begin{enumerate}[resume=p]
\itep
\benu
\item Prove that $K_{p^n}$ is a $p$ group.
\item Use Problem 1 to prove that $|K_{p^n}|=p^n$ and $|K_m|=m$.
\item Use induction on $k$ to prove Part I of the Structure Theorem for Finite Abelian Groups:
\\\\ Let $G$ be an finite abelian group, $p_1<p_2<p_k$ be primes, and $|G|=p_1^{n_1}p_2^{n_2}\dots p_k^{n_k}$ then $G$ is isomorphic to product $G(p_1)\times G(p_2)\times \dots G(p_n)$ where for each $i$, $G(p_i)$ is a p-group of order $p_i^{n_i}$

\eenu
\end{enumerate}


\newpage

\begin{enumerate}[resume=p]
\itep Quoting Lemma 15 of the paper {\em Adkins Finite Groups} (below), prove Theorem \ref{structabelian} part \ref{pt2} (below)
\\\\
\textbf{Lemma:} Let $G$ be a finite abelian p-group, and let $H$ be a cyclic subgroup of maximal order.  Then there is a subgroup $K$ of $G$ such that $G$ is the internal direct product of $H$ and $K$ 
\\\\
\textbf{Theorem:} ??$ 
\end{enumerate}




\end{document}