\documentclass[11pt]{article}
\usepackage{amsmath}
\usepackage{amssymb}
\title{\textbf{Math 311 - Problem Set 1}}
\author{Guy Matz}
\date{\today}
\begin{document}

%\maketitle

%\newpage % Q1

\begin{enumerate}  % begin numbers
\item Fill out the values for md and qt in the following table.\\
\\
\begin{center}
  \begin{tabular}{ c | c | c }
    $(m,n)$ & $md(m,n)$ & $qt(m,n)$ \\ \hline
    (987654321, 7531) & 1326 & 131145 \\ \hline
    (987654321, -7531) & 1326 & -131145 \\ \hline
    (-987654321, 7531) & 6205 & -131146 \\ \hline
    (-987654321, -7531) & 6205 & 131146 \\ \hline
  \end{tabular}
\end{center}

\newpage %Q2
\item Give an example of a set, $S$, such that $S$ is realitively prime, but such that for all $s_i, s_j \in S$, gcd($s_i, s_j) \neq 1$
$$\{6,10,15\}$$

\newpage % Q3
\item Generalizing BB Proposition 1.2.3, which states: \\
Let $a, b$ be integers, where $a \neq 0$ or $b \neq 0$
\begin{enumerate}
\item If $b | ac$, then $b | (a,b) \cdot c$
\item If $b | ac$, and $(a, b) = 1$, then $b|c$
\item If $b | a$, $c | a$ and $(b, c) = 1$, then $bc|a$
\item $(a, bc) =1 $ if and only if $(a,b) = 1$ and $(a,c) = 1$
\end{enumerate}

\begin{enumerate}
\item Generalizing BB Proposition 1.2.3.d, Show that if $(a,b_1)=(a,b_2)=... (a,b_n)=1$ then $(a,b_1 b_2 ... b_n) = 1$\\
\\
\item Show that if $(a, b) = 1$ then $(a^n, b^n) = 1$\\
\\
\item Show that if $(a, b_1 b_2...b_n) = 1$ then $(a, b_1) = (a, b_2) = ... (a,b_n) = 1$\\
\\
\item Generalizing BB Proposition 1.2.3.c, prove that if $a_1 | c$, $a_2| c$...$a_n|c$ and $(a_i, a_j) = 1$ for $i \neq j$ then $(a_1 a_2...a_n)|c$

\end{enumerate}

\newpage % Q4
\item Prove the General Euclid Lemma: Let $p$ be prime and $S$ be a finite set of integers.  If $p|\Pi_{n \in S}n$ then there exists an $n \in S$ such that $p | n$
\\
\\


\newpage % Q5
\item Prove the following:
\begin{enumerate}
\item Let $n \in \mathbb{Z}$ and $2 \leq \mathbb{N}$.  We have $e = e(n,b)$ if and only if $n = b^em$ with $b \nmid m$\\
\\
\item Let $n \in \mathbb{Z}$ and $p$ be prime, then $e = e(n,b)$ if and only if $n = p^em$ with gcd$(p,m) = 1$
\end{enumerate}


\newpage % Q6
\item 
\begin{enumerate}
\item Let $m,n \in \mathbb{Z}^*$ and $d = $ gcd($m,n)$.  Show that gcd($\tfrac{m}{d} , \tfrac{n}{d}) = 1$
\\\\
Since $\lambda m + \omega n = d$, we have $\lambda \tfrac{m}{d} + \omega \tfrac{n}{d} = 1$, which implies gcd($\tfrac{m}{d} , \tfrac{n}{d}) = 1$
\\
\item Suppose that gcd($r,s$) = 1 and gcd($a,b$) = 1 and that $ra = sb$.  Show that $r=b, s = a$ or $r = -b, s = -a$
\\
\item Let $(r,s) = 1$, $a,b \in \mathbb{Z}^*$, $d$ = gcd$(a,b)$ and that $ra = sb$.  Show that $r = \tfrac{b}{d}, s = \tfrac{a}{d}$ or $r = -\tfrac{b}{d}, s=\tfrac{a}{d}$
\\
\item Let $r,s \in \mathbb{Z}_*$, $a,b \in \mathbb{Z}^*$, $d$ = gcd$(a,b)$ and suppose that $ra = sb$.  Show that $\tfrac{r}{f} = \tfrac{b}{d}, \tfrac{s}{f} = \tfrac{a}{d}$ or $\tfrac{r}{f} = -\tfrac{b}{d}, \tfrac{s}{f}=\tfrac{a}{d}$
\end{enumerate}


\newpage % Q7
\item Let $a, b \in \mathbb{Z}^*$ and consider the subgroups $a\mathbb{Z}$ and $a\mathbb{Z}$.  Then $a\mathbb{Z} \cap b\mathbb{Z}$ is a set of common multiples of $a$ and $b$.
\begin{enumerate}
\item Show that $a\mathbb{Z} \cap b\mathbb{Z}$ is a non-trivial subgroup of $\mathbb{Z}$
\\
\item By BB Proposition we have a positive number $l$ such that $a\mathbb{Z} \cap b\mathbb{Z} = l\mathbb{Z}$.  Show that $l$ is the least common multiple of $m$ and $n$.  We will denote this least common multiple by lcm($a,b$) or [$a,b$].
\\
\item Let lcm($a,b) = r|a| = s|b|$.  Let $d$ = gcd($a,b$).  Show that $r = \tfrac{|b|}{d}$ and $s = \tfrac{|a|}{d}$.  That is, lcm$(a,b) = \tfrac{|ab|}{\mathtt{gcd}(a,b)}$
\end{enumerate}


\newpage % Q8
\item Give the steps in the Euclidean algorithm to compute gcd(81665731241, 777695550).

\begin{align*}
81665731241 =& 105 * 777695550 + 7698491\\
777695550 =& 101 * 7698491 + 147959\\
7698491 =& 52 * 147959 + 4623\\
147959 =& 32 * 4623 + 23\\
4623 =& 201 * 23 + 0\\
d =& 23\\
\end{align*}

\newpage % Q9
\item 
\begin{enumerate}
\item Let $r_0 = 8245, r_1 = 2584$.  Show the steps in the extended Euclidean algorithm to get
\begin{equation*}
\left(
\begin{array}{ccc}
s_k & t_k & r_k \\
s_{k+1} & t_{k+1} & 0 \\
\end{array} \right)
\end{equation*}

\begin{equation*}
\left(
\begin{array}{ccc}
 1 & 0 & 8245 \\
 0 & 1 & 2584 \\
\end{array}
\right)\\
\end{equation*}
\begin{equation*}
\left(
\begin{array}{ccc}
 0 & 1 & 2584 \\
 1 & -3 & 493 \\
\end{array}
\right)\\
\end{equation*}
\begin{equation*}
\left(
\begin{array}{ccc}
 1 & -3 & 493 \\
 -5 & 16 & 119 \\
\end{array}
\right)
\end{equation*}
\begin{equation*}
\left(
\begin{array}{ccc}
 -5 & 16 & 119 \\
 21 & -67 & 17 \\
\end{array}
\right)
\end{equation*}\\
\begin{equation*}
\left(
\begin{array}{ccc}
 21 & -67 & 17 \\
 -152 & 485 & 0 \\
\end{array}
\right)\\
\end{equation*}


\item Give the steps to find $(m_{8245}, n_{2584})$.  You Should get (21, -67)
\\
$$(m_{8245}, n_{2584})  = (s_k, t_k) = (21, -67)$$
\\
\item Give the steps to find $(n_{8245}, m_{2584})$.  You Should get (-131, 418)
\\
$$(n_{8245}, m_{2584}) = (21 - \tfrac{2584}{17}, -67 + \tfrac{8245}{17}) = (-131, 418)$$
\end{enumerate}


\newpage % Q10
\item 
\begin{enumerate}
\item Let $r_0 = 79, r_1 = 33$.  Show the steps in the extended Euclidean algorithm to get
\begin{equation*}
\left(
\begin{array}{ccc}
s_k & t_k & r_k \\
s_{k+1} & t_{k+1} & 0 \\
\end{array} \right)
\end{equation*}

\begin{equation*}
\left(
\begin{array}{ccc}
 1 & 0 & 79 \\
 0 & 1 & 33 \\
\end{array}
\right)
\end{equation*}

\begin{equation*}
\left(
\begin{array}{ccc}
 0 & 1 & 33 \\
 1 & -2 & 13 \\
\end{array}
\right)
\end{equation*}

\begin{equation*}
\left(
\begin{array}{ccc}
 1 & -2 & 13 \\
 -2 & 5 & 7 \\
\end{array}
\right)
\end{equation*}

\begin{equation*}
\left(
\begin{array}{ccc}
 -2 & 5 & 7 \\
 3 & -7 & 6 \\
\end{array}
\right)
\end{equation*}

\begin{equation*}
\left(
\begin{array}{ccc}
 3 & -7 & 6 \\
 -5 & 12 & 1 \\
\end{array}
\right)
\end{equation*}

\begin{equation*}
\left(
\begin{array}{ccc}
 -5 & 12 & 1 \\
 33 & -79 & 0 \\
\end{array}
\right)
\end{equation*}



\item Give the steps to find $(m_{79}, n_{33})$.  You Should get (28, -67)
\\
$$(m_{79}, n_{33}) = = (-5 + \tfrac{33}{1}, 12 + \tfrac{-79}{1}) = (28, -67)$$
\item Give the steps to find $(n_{8245}, m_{2584})$.  You Should get (-5, 12)
\\
$$(n_{79}, m_{33}) = (s_k, t_k) = (-5, 12)$$
\end{enumerate}

\newpage % Q11
\item We really are morally obliged to show that these properties are independent of each other. Ideally we would like our sets and relations to reflect common mathematical experiences.
\begin{enumerate}
\item Give an example of a set $S$ and a relation on $\mathcal{R}$ on $s$ which is reflexive and transitive but not symmetric: Hint: Divisibility or $\leq$ or $\subseteq$.
\\
\\
For integers $a, b$ we define $a \sim b$ if $a \leq b$\\
\\Reflexivity: $a \sim a$, $a \leq a \checkmark$
\\Symmetry: $a \sim b \implies b \sim a$, If $a \leq b$ then $b \leq a$. $X$ 
\\Transitivity: $a \sim b \wedge b \sim c \implies a \sim c$, If $a \leq b$ and $b \leq c$, then $a \leq c \checkmark$
\\
\item Give an example of a set $S$ and a relation on $\mathcal{R}$ on $s$ which is reflexive and symmetric but not transitive:\\
\\
For $a,b,c \in \mathbb{R}$ we define $a \sim b$ if $|a - b| \leq \delta$ for some arbitrarily small $\delta \in \mathbb{R}_+$.
\\
\\Reflexivity: $a \sim a$, $|a-a| = 0 \leq \delta \checkmark$
\\Symmetry: $a \sim b \implies b \sim a$, $|a-b| = |b-a|$, so if $|a-b| \leq \delta$, then $|b-a| \leq \delta  \checkmark$
\\Transitivity: $a \sim b \wedge b \sim c \implies a \sim c$, This fails since, e,g, with $a = 5, b = 4, c = 3, |a-b| = 1$, and $|b-c| = 1$, however $|a-c| = 2$
\item Show that if $\mathcal{R}$ on $S$ is symmetric and transitive but not reflexive there must be an $a \in S$ such that for all $b \in S$, $a /\mathcal{R} b$ and $b  /\mathcal{R} a$. In terms of set theory this is $\{a\} \times S \cup S \times \{a\} \cap \mathcal{R} = \phi$. In this case we say that $a$ is isolated with respect to $\mathcal{R}$. Now give an example of a $S$ and a relation $\mathcal{R}$ on $S$ which is symmetric, transitive but not reflexive.
\\
\\
Example: For integers $a,b$ we define $a \sim b$ if $|a-b| > 0$
\end{enumerate}

\newpage % Q12
\item Prove Theorem 44. Each proof should be one or two lines which are references to the results in our notes on finite sets and counting. You should refer to this document by NFC.
\begin{enumerate}
\item If $S$ is finite and $P$ be a partition of $S$ then $P$ is finite and for all $s \in S, [s]^P$ is finite and $|S| = \Sigma_A \in P|A|$.
\\
\item If $P$ is a finite partition of $S$ into finite sets then $S$ is finite and $|S| = \Sigma_{A \in P}|A|$.
\end{enumerate}

\newpage % Q13
\item Prove Theorem 50: Let $S \subseteq \mathbb{Z}$ and define a relation $\sim_S$ on $\mathbb{Z}$ by $a \sim_S b$ if $a-b \in S$.  The relation $\sim_S$ is an equivalence relation on $\mathbb{Z}$ if and only if $S$ is a subgroup of $\mathbb{Z}$.

\newpage % Q14
\item Let $P$ be an additive partition of $\mathbb{Z}$. Show that $[0]^P$ is a subgroup of $P$.


\newpage % Q15
\item
\begin{enumerate}
\item Let $G$ be a subgroup of $\mathbb{Z}$. By Theorem 50, $\sim_G$ is an equivalence relation. Show that $[a]_{\sim_G} = \{a\}+G = ???$ . Notice that $\{a\} + G = \{a + g : g \in S\}$
\\
\item Show that if G is a subgroup of $\mathbb{Z}$ then $\mathbb{Z}\sim_G$ is additive and that $[a]\sim_G + [b]\sim_G = [a + b]\sim_G$.
\end{enumerate}


\end{enumerate} % end numbers
\end{document}
