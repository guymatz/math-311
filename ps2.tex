\documentclass[11pt]{article}
\usepackage{amsmath}
\usepackage{amssymb}
\title{\textbf{Math 311 - Problem Set 1}}
\author{Guy Matz}
\date{\today}
\begin{document}

%\maketitle

%\newpage % Q1

\begin{enumerate}  % begin numbers
\item Fill out the values for md and qt in the following table.\\
\\
\begin{center}
  \begin{tabular}{ c | c | c }
    $(m,n)$ & $md(m,n)$ & $qt(m,n)$ \\ \hline
    (987654321, 7531) & 1326 & 131145 \\ \hline
    (987654321, -7531) & 1326 & -131145 \\ \hline
    (-987654321, 7531) & 6205 & -131146 \\ \hline
    (-987654321, -7531) & 6205 & 131146 \\ \hline
  \end{tabular}
\end{center}

\newpage %Q2
\item Give an example of a set, S$S$, such that $S$ is realitively prime, but such that for all $s_i, s_j \in S$, gcd($s_i, s_j) \neq 1$
$$\{6,10,15\}$$

\newpage % Q3
\item Generalizing BB Proposition 1.2.3, which states: \\
Let $a, b$ be integers, where $a \neq 0$ or $b \neq 0$
\begin{enumerate}
\item If $b | ac$, then $b | (a,b) \cdot c$
\item If $b | ac$, and $(a, b) = 1$, then $b|c$
\item If $b | a$, $c | a$ and $(b, c) = 1$, then $bc|a$
\item $(a, bc) =1 $ if and only if $(a,b) = 1$ and $(a,c) = 1$
\end{enumerate}

\begin{enumerate}
\item Generalizing BB Proposition 1.2.3.d, Show that if $(a,b_1)=(a,b_2)=... (a,b_n)=1$ then $(a,b_1 b_2 ... b_n) = 1$\\
\\
\item Show that if $(a, b) = 1$ then $(a^n, b^n) = 1$\\
\\
\item Show that if $(a, b_1 b_2...b_n) = 1$ then $(a, b_1) = (a, b_2) = ... (a,b_n) = 1$\\
\\
\item Generalizing BB Proposition 1.2.3.c, prove that if $a_1 | c$, $a_2| c$...$a_n|c$ and $(a_i, a_j) = 1$ for $i \neq j$ then $(a_1 a_2...a_n)|c$

\end{enumerate}

\newpage % Q4
\item Prove the General Euclid Lemma: Let $p$ be prime and $S$ be a finite set of integers.  If $p|\Pi_{n \in S}n$ then there exists an $n \in S$ such that $p | n$
\\
\\


\newpage % Q5
\item Prove the following:
\begin{enumerate}
\item Let $n \in \mathbb{Z}$ and $2 \leq \mathbb{N}$.  We have $e = e(n,b)$ if and only if $n = b^em$ with $b \nmid m$\\
\\
\item Let $n \in \mathbb{Z}$ and $p$ be prime, then $e = e(n,b)$ if and only if $n = p^em$ with gcd$(p,m) = 1$
\end{enumerate}


\newpage % Q6
\item 
\begin{enumerate}
\item Let $m,n \in \mathbb{Z}^*$ and $d = $ gcd($m,n)$.  Show that gcd($m/d , n/d) = 1$
\\
\item Suppose that gcd($r,s$) = 1 and gcd($a,b$) = 1 and that $ra = sb$.  Show that $r=b, s = a$ or $r = -b, s = -a$
\\
\item Let $(r,s) = 1$, $a,b \in \mathbb{Z}^*$, $d$ = gcd$(a,b)$ and that $ra = sb$.  Show that $r = \tfrac{b}{d}, s = \tfrac{a}{d}$ or $r = -\tfrac{b}{d}, s=\tfrac{a}{d}$
\\
\item Let $r,s \in \mathbb{Z}_*$, $a,b \in \mathbb{Z}^*$, $d$ = gcd$(a,b)$ and suppose that $ra = sb$.  Show that $\tfrac{r}{f} = \tfrac{b}{d}, \tfrac{s}{f} = \tfrac{a}{d}$ or $\tfrac{r}{f} = -\tfrac{b}{d}, \tfrac{s}{f}=\tfrac{a}{d}$
\end{enumerate}


\newpage % Q7
\item Let $a, b \in \mathbb{Z}^*$ and consider the subgroups $a\mathbb{Z}$ and $a\mathbb{Z}$.  Then $a\mathbb{Z} \cap b\mathbb{Z}$ is a set of common multiples of $a$ and $b$.
\begin{enumerate}
\item Show that $a\mathbb{Z} \cap b\mathbb{Z}$ is a non-trivial subgroup of $\mathbb{Z}$
\\
\item By BB Proposition we have a positive number $l$ such that $a\mathbb{Z} \cap b\mathbb{Z} = l\mathbb{Z}$.  Show that $l$ is the least common multiple of $m$ and $n$.  We will denote this least common multiple by lcm($a,b$) or [$a,b$].
\\
\item Let lcm($a,b) = r|a| = s|b|$.  Let $d$ = gcd($a,b$).  Show that $r = \tfrac{|b|}{d}$ and $s = \tfrac{|a|}{d}$.  That is, lcm$(a,b) = \tfrac{|ab|}{\mathtt{gcd}(a,b)}$
\end{enumerate}


\newpage % Q8
\item Give the steps in the Euclidean algorithm to compute gcd(81665731241, 777695550).

\begin{align*}
81665731241 =& 105 * 777695550 + 7698491\\
777695550 =& 101 * 7698491 + 147959\\
7698491 =& 52 * 147959 + 4623\\
147959 =& 32 * 4623 + 23\\
4623 =& 201 * 23 + 0\\
d =& 23\\
\end{align*}

\end{enumerate} % end numbers
\end{document}
